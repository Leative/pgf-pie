% SPDX-License-Identifier: GPL-2.0-only OR LPPL-1.3c
\RequirePackage{scalefnt}

%
% Utilities
%

\def\pgfpie@ifx#1#2{%
  \ifx#1#2%
    \expandafter\pgfutil@firstoftwo
  \else
    \expandafter\pgfutil@secondoftwo
  \fi}

\def\pgfpie@ifdim#1#2#3{%
  \ifdim\dimexpr#1\relax#2\dimexpr#3\relax
    \expandafter\pgfutil@firstoftwo
  \else
    \expandafter\pgfutil@secondoftwo
  \fi}

\def\pgfpie@ifnum#1#2#3{%
  \ifnum\numexpr#1\relax#2\numexpr#3\relax
    \expandafter\pgfutil@firstoftwo
  \else
    \expandafter\pgfutil@secondoftwo
  \fi}

\def\pgfpie@ifodd#1{%
  \ifodd\numexpr#1\relax
    \expandafter\pgfutil@firstoftwo
  \else
    \expandafter\pgfutil@secondoftwo
  \fi}

\def\pgfpie@newif#1{%
  \csname newif\expandafter\endcsname\csname ifpgfpie@#1\endcsname
  \expandafter\def\csname pgfpie@if#1\expandafter\endcsname\expandafter{%
    \csname ifpgfpie@#1\endcsname
      \expandafter\pgfutil@firstoftwo
    \else
      \expandafter\pgfutil@secondoftwo
    \fi}%
}

%
% Constants for text comparison
%

\def\pgfpie@text@inside{inside}
\def\pgfpie@text@legend{legend}
\def\pgfpie@sum@c{100}
\def\pgfpie@sum@auto{auto}

% args:
% #1: begin angle
% #2: end angle
% #3: number
% #4: label
% #5: explode
% #6: fill color
% #7: radius
% #8: center
\def\pgfpie@slice#1#2#3#4#5#6#7#8{%
  \pgfmathparse{0.5*(#1)+0.5*(#2)}
  \let\pgfpie@midangle\pgfmathresult

  \path (#8) -- ++({\pgfpie@midangle}:{#5}) coordinate (pgfpie@O);

  \pgfmathparse{(#7)+(#5)}
  \let\pgfpie@radius\pgfmathresult
  
  % slice
  \draw[line join=round,fill={#6},\pgfpie@style] (pgfpie@O) -- ++({#1}:{#7}) arc ({#1}:{#2}:{#7}) -- cycle;

  \pgfmathparse{min(((#2)-(#1)-10)/110*(-0.3),0)}
  \pgfmathparse{(max(\pgfmathresult,-0.5) + 0.8)*(#7)}
  \let\pgfpie@innerpos\pgfmathresult

  \pgfpie@ifx\pgfpie@text\pgfpie@text@inside{%
    % label and number together
    \path (pgfpie@O) -- ++({\pgfpie@midangle}:{\pgfpie@innerpos}) node[align=center]
    {\pgfpie@scalefont{#3}\pgfpie@labeltext{#4}\\\pgfpie@numbertext{#3}};
  }{%
    % label
    \pgfpie@ifhidelabel{}{%
      \pgfpie@iflegend{}{%
        \path (pgfpie@O) -- ++ ({\pgfpie@midangle}:{\pgfpie@radius})
        node[inner sep=0, \pgfpie@text={\pgfpie@midangle:#4}]{};
      }%
    }%

    % number
    \path (pgfpie@O) -- ++({\pgfpie@midangle}:{\pgfpie@innerpos}) node
    {\pgfpie@scalefont{#3}\pgfpie@numbertext{#3}};
  }%
}

\def\pgfpie@findColor#1{%
  \pgfmathparse{int(mod(#1,\the\pgfpie@colorLength))}%
  \let\pgfpie@ci\pgfmathresult
  \foreach \pgfpie@c [count=\pgfpie@j from 0] in \pgfpie@color {%
    \pgfpie@ifnum{\pgfpie@j}{=}{\pgfpie@ci}{%
      \xdef\pgfpie@thecolor{\pgfpie@c}%
      \breakforeach
    }{}%
  }
}

\def\pgfpie@findExplode#1{%
  \pgfmathparse{int(mod(#1,\the\pgfpie@explodeLength))}%
  \let\pgfpie@ei\pgfmathresult
  \foreach \pgfpie@e [count=\pgfpie@j from 0] in \pgfpie@explode {%
    \pgfpie@ifnum{\pgfpie@j}{=}{\pgfpie@ei}{%
      \xdef\pgfpie@theexplode{\pgfpie@e}%
      \breakforeach
    }{}%
  }%
}

% #1: bottom left point
% #2: size
% #3: number
% #4: color
% #5: text
\def\pgfpie@square#1#2#3#4#5{%
  \pgfpie@ifx\pgfpie@text\pgfpie@text@inside{%
    \draw[fill={#4},\pgfpie@style] (#1) rectangle node[align=center]
    {\pgfpie@scalefont{#3}\pgfpie@labeltext{#5}\\\pgfpie@numbertext{#3}} ++(#2);
  }{%
    \draw[fill={#4},\pgfpie@style] (#1) rectangle node
    {\pgfpie@scalefont{#3}\pgfpie@numbertext{#3}} ++(#2);
  }%
}

% #1: pos
% #2: radius
% #3: number
% #4: color
% $5: style
% $6: label
\def\pgfpie@cloud#1#2#3#4#5#6{%
  \draw[fill=#4, #5] (#1) circle[radius=#2];
  \pgfpie@ifx\pgfpie@text\pgfpie@text@inside{%
    \node[align=center] at (#1) {\pgfpie@scalefont{#3}\pgfpie@labeltext{#6}\\\pgfpie@numbertext{#3}};
  }{%
    \node at (#1) {\pgfpie@scalefont{#3}\pgfpie@numbertext{#3}};
  }%
}

\newdimen\pgfpie@angleEnd
\newcount\pgfpie@explodeLength
\newcount\pgfpie@colorLength
\newcount\pgfpie@sliceLength

\pgfqkeys{/pgfpie}{%
  .search also={/tikz,/pgf},
  explode/.store in=\pgfpie@explode,
  color/.store in=\pgfpie@color,
  radius/.store in=\pgfpie@radius,
  pos/.store in=\pgfpie@pos,
  style/.store in=\pgfpie@style,
  before number/.store in=\pgfpie@beforenumber,
  after number/.code={%
    \def\pgfpie@afternumber{#1}%
    \def\pgfpie@late@afternumber{}%
  },
  text/.store in=\pgfpie@text,
  sum/.code={%
    \def\pgfpie@sum{#1}%
    \pgfpie@ifx\pgfpie@sum\pgfpie@sum@c}%
    }{%
      \def\pgfpie@late@afternumber{}%
    }%
  },
  rotate/.store in=\pgfpie@rotate,
}

\pgfpie@newif{polar}
\pgfqkeys{/pgfpie}{polar/.is if=pgfpie@polar}

\pgfpie@newif{legend}

\pgfpie@newif{square}
\pgfqkeys{/pgfpie}{square/.is if=pgfpie@square}

\pgfpie@newif{cloud}
\pgfqkeys{/pgfpie}{cloud/.is if=pgfpie@cloud}

\pgfpie@newif{scalefont}
\pgfqkeys{/pgfpie}{scale font/.is if=pgfpie@scalefont}
\def\pgfpie@scalefont#1{%
  \pgfpie@ifscalefont{%
    \pgfmathparse{(#1) / (\pgfpie@sum) * 3 + 0.9}%
    \scalefont{\pgfmathresult}%
  }{}%
}

\pgfpie@newif{hidenumber}
\pgfqkeys{/pgfpie}{hide number/.is if=pgfpie@hidenumber}
\def\pgfpie@numbertext#1{%
  \pgfpie@ifhidenumber{}{%
    \pgfpie@beforenumber#1\pgfpie@afternumber
  }%
}

\pgfpie@newif{hidelabel}
\pgfqkeys{/pgfpie}{hide label/.is if=pgfpie@hidelabel}
\def\pgfpie@labeltext#1{\pgfpie@ifhidelabel{}{#1}}

\tikzaddtikzonlycommandshortcutlet{\pie}{\pgfpie@pie}%

\def\pgfpie@pie{%
  \pgfutil@ifnextchar[{\pgfpie@@pie}{\pgfpie@@pie[]}%
}

\def\pgfpie@@pie[#1]#2{%
  \scope[%
    % to be compatible with "babel" lib
    handle active characters in nodes=false,
    % load default parameters
    /pgfpie/.cd,
    explode=0,
    color={blue!60, cyan!60, yellow!60, orange!60, red!60,
      blue!60!cyan!60, cyan!60!yellow!60, red!60!cyan!60,
      red!60!blue!60, orange!60!cyan!60},
    radius=3,
    pos={0,0},
    style={thick},
    before number=,
    after number=,
    text=label,
    sum=100,
    rotate=0,
    polar=false,
    square=false,
    cloud=false,
    scale font=false,
    hide number=false,
    hide label=false,
    every pie/.try,
    % load user's parameters
    #1]

  % add percentage automatically
  \pgfpie@late@afternumber

  % legend or not
  \pgfpie@ifx\pgfpie@text\pgfpie@text@legend{%
    \pgfpie@legendtrue
  }{%
    \pgfpie@legendfalse
  }%

  % handle sum
  \pgfpie@ifx\pgfpie@sum\pgfpie@sum@auto{%
    % sum all input
    \xdef\pgfpie@sum{0}%
    \foreach \pgfpie@p/\pgfpie@t in {#2} {%
      \pgfmathparse{(\pgfpie@sum) + (\pgfpie@p)}%
      \xdef\pgfpie@sum{\pgfmathresult}%
    }%
  }{}%

  % init counters
  \global\pgfpie@explodeLength=0
  \foreach \pgfpie@e in \pgfpie@explode {%
    \global\advance\pgfpie@explodeLength by 1
  }%

  \global\pgfpie@colorLength=0
  \foreach \pgfpie@c in \pgfpie@color {%
    \global\advance\pgfpie@colorLength by 1
  }%
  
  \pgfmathsetlength{\pgfpie@angleEnd}{0}%

  \global\pgfpie@sliceLength=0
  \foreach \pgfpie@p/\pgfpie@e in {#2} {%
    \global\advance\pgfpie@sliceLength by 1
  }%

  \pgfpie@ifsquare{%
    \pgfpie@pie@square{#2}%
    }{\pgfpie@ifcloud{\pgfpie@pie@cloud{#2}%
      }{\pgfpie@pie@circle{#2}}}

  % legend
  \pgfpie@ifhidelabel{}{%
    \pgfpie@iflegend{%
      \pgfpie@legend{#2}%
  }{}}%
  \endscope
}

\def\pgfpie@pie@square#1{%
  \pgfmathparse{(\pgfpie@radius)*2}
  \xdef\pgfpie@verticalLength{\pgfmathresult}
  \xdef\pgfpie@horizontalLength{\pgfmathresult}
  \path (\pgfpie@pos) -- ++({-\pgfpie@radius},{-\pgfpie@radius}) coordinate (pgfpie@start);
  \pgfmathparse{(\pgfpie@verticalLength) * (\pgfpie@horizontalLength) / (\pgfpie@sum)}
  \let\pgfpie@squareUnit\pgfmathresult

  % drawing loop
  \foreach \pgfpie@p/\pgfpie@t [count=\pgfpie@i from 0] in {#1}
  {
    \pgfpie@findColor{\pgfpie@i}

    \pgfpie@ifdim{\pgfpie@verticalLength cm}{>}{\pgfpie@horizontalLength cm}{%
      \pgfmathparse{(\pgfpie@p) * (\pgfpie@squareUnit) / (\pgfpie@horizontalLength)}
      \let\pgfpie@height\pgfmathresult
      
      \pgfpie@square{pgfpie@start}{\pgfpie@horizontalLength,\pgfpie@height}
      {\pgfpie@p}
      {\pgfpie@thecolor}
      {\pgfpie@t}
      %label
      \pgfpie@iflegend{}{%
        \pgfpie@ifx\pgfpie@text\pgfpie@text@inside{}{%
          \pgfpie@ifhidelabel{}{%
            \path (pgfpie@start) -- ++({\pgfpie@horizontalLength},{\pgfpie@height*0.5})
            node[inner sep=0, \pgfpie@text={0:{\pgfpie@t}}]{};
          }%
        }%
      }%

      \pgfmathparse{\pgfpie@verticalLength - \pgfpie@height}
      \xdef\pgfpie@verticalLength{\pgfmathresult}
      \path (pgfpie@start) -- ++(0, \pgfpie@height) coordinate (pgfpie@start);
    }{%
      \pgfmathparse{(\pgfpie@p) * (\pgfpie@squareUnit) / (\pgfpie@verticalLength)}%
      \let\pgfpie@width\pgfmathresult

      \pgfpie@square{pgfpie@start}{\pgfpie@width,\pgfpie@verticalLength}
      {\pgfpie@p}
      {\pgfpie@thecolor}
      {\pgfpie@t}
      %label
      \pgfpie@iflegend{}{%
        \pgfpie@ifx\pgfpie@text\pgfpie@text@inside{}{%
          \pgfpie@ifhidelabel{}{%
            \path (pgfpie@start) -- ++({\pgfpie@width*0.5},{\pgfpie@verticalLength})
            node[inner sep=0, \pgfpie@text={90:{\pgfpie@t}}]{};
          }%
        }%
      }%

      \pgfmathparse{(\pgfpie@horizontalLength) - (\pgfpie@width)}%
      \xdef\pgfpie@horizontalLength{\pgfmathresult}%
      \path (pgfpie@start) -- ++({\pgfpie@width},0) coordinate (pgfpie@start);
    }%
  }%
}

\def\pgfpie@pie@cloud#1{%
  % drawing loop
  \foreach \pgfpie@p/\pgfpie@t [count=\pgfpie@i from 0] in {#1}
  {
    % find explode
    \pgfpie@findExplode{\pgfpie@i}
    \def\pgfpie@cloudGap{(\pgfpie@theexplode) + 0.1}
    \pgfmathparse{sqrt((\pgfpie@p) / (\pgfpie@sum)) * (\pgfpie@radius)}
    \let\pgfpie@cloudR\pgfmathresult
    \pgfpie@ifnum{\pgfpie@i}{=}{0}{%
      % first cloud
      \coordinate (pgfpie@O) at (\pgfpie@pos);
      \xdef\pgfpie@cloudRone{\pgfpie@cloudR}
      \xdef\pgfpie@cloudExtendDir{180+(\pgfpie@rotate)}
    }{\pgfpie@ifnum{\pgfpie@i}{=}{1}{%
      % second cloud
      \xdef\pgfpie@cloudRtwo{\pgfpie@cloudR}
      \xdef\pgfpie@cloudExtendDir{45+(\pgfpie@rotate)}
      \path (pgfpie@O) -- ++({\pgfpie@cloudExtendDir}:{(\pgfpie@cloudRone)+(\pgfpie@cloudGap)+(\pgfpie@cloudRtwo)}) coordinate (pgfpie@O);
    }{%
      % next cloud
      \pgfmathparse{(\pgfpie@cloudRone)+(\pgfpie@cloudGap)+(\pgfpie@cloudRtwo)}
      \let\pgfpie@la\pgfmathresult
      \pgfmathparse{(\pgfpie@cloudRone)+(\pgfpie@cloudGap)+(\pgfpie@cloudR)}
      \let\pgfpie@lb\pgfmathresult
      \pgfmathparse{(\pgfpie@cloudRtwo)+(\pgfpie@cloudGap)+(\pgfpie@cloudR)}
      \let\pgfpie@lc\pgfmathresult
      \pgfmathparse{(\pgfpie@la)^2+(\pgfpie@lc)^2-(\pgfpie@lb)^2}
      \let\pgfpie@tmp\pgfmathresult
      \pgfmathparse{180 - acos((\pgfpie@tmp) / 2 / (\pgfpie@la) / (\pgfpie@lc))}
      \let\pgfpie@cloudRot\pgfmathresult
      \pgfpie@ifodd{\pgfpie@i}{%
        \pgfmathparse{-(\pgfpie@cloudRot)}
        \let\pgfpie@cloudRot\pgfmathresult
      }{}%
      \pgfmathparse{(\pgfpie@cloudExtendDir) - (\pgfpie@cloudRot)}
      \xdef\pgfpie@cloudExtendDir{\pgfmathresult}
      \path (pgfpie@O) -- ++({\pgfpie@cloudExtendDir}:{\pgfpie@lc}) coordinate (pgfpie@O);
      \xdef\pgfpie@cloudRone{\pgfpie@cloudRtwo}
      \xdef\pgfpie@cloudRtwo{\pgfpie@cloudR}
    }}%

    % find color
    \pgfpie@findColor{\pgfpie@i}

    \pgfpie@cloud{pgfpie@O}{\pgfpie@cloudR}{\pgfpie@p}
    {\pgfpie@thecolor}{\pgfpie@style}{\pgfpie@t}

    % label
    \pgfpie@iflegend{}{%
      \pgfpie@ifx\pgfpie@text\pgfpie@text@inside{}{%
        \pgfpie@ifhidelabel{}{%
          \path (pgfpie@O) -- ++({\pgfpie@cloudExtendDir}:{\pgfpie@cloudR})
          node[inner sep=0, \pgfpie@text={\pgfpie@cloudExtendDir:{\pgfpie@t}}] {};
        }%
      }%
    }%
  }%
}

\def\pgfpie@pie@circle#1{%
  \pgfpie@ifpolar{%
    \xdef\pgfpie@maxValue{0}
    \foreach \pgfpie@p/\pgfpie@e in {#1} {
      \pgfmathparse{\pgfpie@p}%
      \pgfpie@ifdim{\pgfpie@maxValue pt}{<}{\pgfmathresult pt}{%
        \xdef\pgfpie@maxValue{\pgfpie@p}
      }{}%
    }
    \pgfmathparse{(\pgfpie@sum) / \the\pgfpie@sliceLength}
    \xdef\pgfpie@polarangle{\pgfmathresult}
    \pgfmathparse{(\pgfpie@radius) / sqrt(\pgfpie@maxValue)}
    \xdef\pgfpie@polarRadiusUnit{\pgfmathresult}
  }{%
    \xdef\pgfpie@theradius{\pgfpie@radius}%
  }

  \xdef\pgfpie@angleBegin{\the\pgfpie@angleEnd}
  % drawing loop
  \foreach \pgfpie@p/\pgfpie@t [count=\pgfpie@i from 0] in {#1}
  {
    \pgfmathsetlength{\pgfpie@angleEnd}{\pgfpie@angleBegin}
    \pgfpie@ifpolar{%
      % Polar area diagram
      \pgfmathaddtolength{\pgfpie@angleEnd}{\pgfpie@polarangle}
      \pgfmathparse{sqrt(\pgfpie@p) * (\pgfpie@polarRadiusUnit)}
      \xdef\pgfpie@theradius{\pgfmathresult}
    }{%
    % normal pie
    \pgfmathaddtolength{\pgfpie@angleEnd}{\pgfpie@p}
    }

    % find explode
    \pgfpie@findExplode{\pgfpie@i}

    % find color
    \pgfpie@findColor{\pgfpie@i}
    \pgfpie@slice{(\pgfpie@angleBegin)/(\pgfpie@sum)*360+(\pgfpie@rotate)}
    {\the\pgfpie@angleEnd/(\pgfpie@sum)*360+(\pgfpie@rotate)}
    {\pgfpie@p}
    {\pgfpie@t}
    {\pgfpie@theexplode}
    {\pgfpie@thecolor}
    {\pgfpie@theradius}
    {\pgfpie@pos}
    \xdef\pgfpie@angleBegin{\the\pgfpie@angleEnd}
  }
}

\def\pgfpie@legend#1{%
  \coordinate[xshift=0.8cm,
  yshift={(\the\pgfpie@sliceLength*0.5+1)*0.5cm}] (pgfpie@legendpos) at
  (current bounding box.east);

  \scope[node distance=0.5cm]
    \foreach \pgfpie@p/\pgfpie@t [count=\pgfpie@i from 0] in {#1}
    {
      \pgfpie@findColor{\pgfpie@i}
      \node[draw, fill={\pgfpie@thecolor}, \pgfpie@style, below of={pgfpie@legendpos}, label={0:{\pgfpie@t}}] (pgfpie@legendpos) {};
    }
  \endscope
}
