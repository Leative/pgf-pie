% Manual of pgf-pie.sty, a convenient set of macros for drawing pie
% chart. Written by Xu Yuan <xuyuan.cn@gmail.com> This file is part of
% pgf-pie you may get it at http://code.google.com/p/pgf-pie/

\documentclass{article}
\usepackage[margin=12mm]{geometry}
\usepackage{hyperref}

\usepackage{pgf-pie}

%%%%%%%%%%%%%%%%%%%%%%%%%%%%%%%%%%%%%%%%%%%%%%%%%%%%%%%%%%%%%%%%%
\usepackage{listings}
\usepackage{color}
\definecolor{listinggray}{gray}{0.92}
\lstset{ %
language=[LaTeX]TeX,
breaklines=true,
frame=single,
% frameround=tttt,
basicstyle=\footnotesize\ttfamily,
backgroundcolor=\color{listinggray},
keywordstyle=\color{blue}
}
%%%%%%%%%%%%%%%%%%%%%%%%%%%%%%%%%%%%%%%%%%%%%%%%%%%%%%%%%%%%%%%%%

%%%%%%%%%%%%%%%%%%%%%%%%%%%%%%%%%%%%%%%%%%%%%%%%%%%%%%%%%%%%%%%%%
\hypersetup{
  colorlinks=true,
  linkcolor=blue,
  anchorcolor=black,
  citecolor=olive,
  filecolor=magenta,
  menucolor=red,
  urlcolor=blue
}
%%%%%%%%%%%%%%%%%%%%%%%%%%%%%%%%%%%%%%%%%%%%%%%%%%%%%%%%%%%%%%%%%

%%%%%%%%%%%%%%%%%%%%%%%%%%%%%%%%%%%%%%%%%%%%%%%%%%%%%%%%%%%%%%%%%
\newcommand{\demo}[2][1]{
\begin{minipage}{.49\linewidth}
\centering
\resizebox{#1\linewidth}{!}{
\input{demo/#2}
}
\end{minipage}
\hspace{0.01\linewidth}
\begin{minipage}{.5\linewidth}
\lstinputlisting{demo/#2}
\end{minipage}
}
%%%%%%%%%%%%%%%%%%%%%%%%%%%%%%%%%%%%%%%%%%%%%%%%%%%%%%%%%%%%%%%%%

%%%%%%%%%%%%%%%%%%%%%%%%%%%%%%%%%%%%%%%%%%%%%%%%%%%%%%%%%%%%%%%%%
\newcommand{\example}[1]{
\resizebox{\linewidth}{!}{
\input{demo/#1}
}
\lstinputlisting{demo/#1}
}
%%%%%%%%%%%%%%%%%%%%%%%%%%%%%%%%%%%%%%%%%%%%%%%%%%%%%%%%%%%%%%%%%

\begin{document}
%%%%%%%%%%%%%%%%%%%%%%%%%%%%%%%%%%%%%%%%%%%%%%%%%%%%%%%%%%%%%%%%%
\title{Drawing Pie Chart by using \texttt{pgf-pie}}
\author{\href{mailto:xuyuan.cn@gmail.com}{Yuan Xu}}
\date{\today{}~(v0.1)}
\maketitle
%%%%%%%%%%%%%%%%%%%%%%%%%%%%%%%%%%%%%%%%%%%%%%%%%%%%%%%%%%%%%%%%%

\begin{abstract}
  \texttt{pgf-pie} is a LaTeX package for drawing pie chart. As stated
  by its name, it is based on a very popular graphic package
  \texttt{PGF/TikZ}. This document presents the usage of
  \texttt{pgf-pie} and collects some pie charts as examples.
  \texttt{pgf-pie} can be downloaded from
  \href{http://code.google.com/p/pgf-pie/}{http://code.google.com/p/pgf-pie/}.
\end{abstract}

\tableofcontents

\section{The Essentials}

\subsection{First Pie}
\demo[0.6]{first-pie}

\subsection{Explode}
\demo[0.6]{explode}

\subsection{Color}
\demo[0.6]{color}

\section{Examples}



% \pie{20/, 4/, 11/, 49/, 16/}

% \pie{20/type A, 4/type B, 11/type C, 49/type D, 16/other}

% \pie[explode={0, 0.01, 0},]{20/type A, 4/type B, 11/type C, 49/type D, 16/other}

% \pie[explode={0, 0.01}, colormap=black!10]{20/type A, 4/type B,
%   11/type C, 49/type D, 16/other}

% \pie{10/, 10/, 10/, 10/, 10/, 10/, 10/, 10/, 10/, 10/ }

% \section{Acknowledgements}
% Many people contributed to \texttt{pgf-pie} by reporting problems,
% suggesting various improvements or submitting code. Here is a list of
% these people:
% \href{mailto:???}{name}.

\end{document}
%%% Local Variables: 
%%% mode: Tex-PDF
%%% TeX-master: t
%%% End: 
